Human engineering prowess is not void of failures. It is generally understood
that there is no exploration without enhanced risk, however; sound engineering
warrants the proper evaluation of possible risks in order to contain potential
losses. The Tacoma Narrows collapse in 1940
\cite{billah-scanian-resonance-tacoma-narrows-and-phys-textbooks} and the
Kansas City walkways collapse in 1981\cite{kansas-city-walkways-collapse}
serve as reminders of what we stand to loose when our structures fail.

Petroski stated about engineers ``For all of their efforts are to one
end: to make something stand that has not stood before, to reassemble Nature
into something new, and above all to obviate failure in the effort''
\cite{petroski-to-eng-is-human}.

In an effort to truly obviate failure one must understand, honor and apply
fundamental engineering principles to developed designs, hence this guide.

Strength and stability are two core issues in terms of understanding
structures. Strength represents the capacity of individual elements to
withstand applied forces, whereas stability represents the capability of a
system to transfer various forces to the ground
\cite{mit-architectonics-chris-luebkeman}.

% TODO: Determine whether the next section is too trivial to be stated.
It is imporant to understand that it is possible to design a system that offers
the required strength on all comprised elements without offering the required
stability. Stability is more a matter of interplay between constituing system
elements.
