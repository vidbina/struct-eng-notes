\subsection{Wood Beams}

\subsubsection{Flexular Stress}

Considering wood as a fibre composite, the flexular strength may be defined as
the stress on the surface of a specimen at a failure, accompanied by the
fracturing of fibers
\cite[chapter 7.2]{hodgikinson-mech-testing-fibre-composites}.

For a bending moment $M$, a width of $b$ and a height of $h$ one may define
the maximum normal stress $\sigma$ as presented in equation
\ref{eq:max-normal-stress}.

\begin{equation}\label{eq:max-normal-stress}
\sigma = \frac{6M}{bh^2}
\end{equation}

With $F_s$ representing the shear force on the cross-section, the maximum shear
stress is presented in equation \ref{eq:max-shear-stress}.

\begin{equation}\label{eq:max-shear-stress}
\tau = \frac{3F_s}{2bh}
\end{equation}

The allowable flexure stress $F'_b$ of wood is tabulated in catalogs, taken
duration $C_D$, moisture $C_M$, beam stablity $C_L$, size $C_F$,
flat use $C_{fu}$ and repetitive member $C_r$ factors into account to define
the flexure as a function of those factors.

\begin{equation}
F'_{b} = F_b(\mathbf C)
\end{equation}

The actual flexure stress is on a wooden beam can be calculated by:

\begin{equation}
f_b = \frac{Mc}{I} = \frac{M}{S} \Biggm\lvert S = \frac{I}{c} = \frac{bd^2}{6}
\end{equation}

\subsubsection{Shear Stress}
The allowable shear stress
